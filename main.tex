\documentclass[10pt, a4paper]{article}
\usepackage[a4paper, left=1in, right=1in, top=1in, bottom=1in]{geometry}
\usepackage{graphicx}
\usepackage{mdframed}
\usepackage{multicol}
\usepackage{listings}
\usepackage{xcolor} 
\usepackage{amsmath}

\usepackage{fancyhdr}           
\pagestyle{fancy}   

\fancyhf{}

\fancyhead[L]{Military Institute of Science and Technology MIST\_EaglesExpr} 
\fancyhead[R]{\thepage} 

\lstdefinestyle{shared}{
    belowcaptionskip=1\baselineskip,
    breaklines=true, 
    showstringspaces=false,
    basicstyle=\ttfamily\small,
    tabsize=2,
}
\lstdefinestyle{cpp}{
    style=shared,
    language=C++,
    commentstyle=\itshape\color{green!50!black},
    keywordstyle=\bfseries\color{blue},
    stringstyle=\color{orange},
    numberstyle=\tiny\color{gray},
}
\lstdefinestyle{java}{
    style=shared,
    language=Java,
    keywordstyle=\bfseries\color{red!22!green!28!blue!50!},
    commentstyle=\itshape\color{black!60!},
    identifierstyle=\bfseries\color{red!19!green!42!blue!39!},
    stringstyle=\bfseries\color{blue!40!green!20!violet!40!},
}
\lstdefinestyle{py}{
    style=shared,
    language=Python,
    keywordstyle=\bfseries\color{red!22!green!28!blue!50!},
    commentstyle=\itshape\color{black!60!},
    identifierstyle=\bfseries\color{red!19!green!42!blue!39!},
    stringstyle=\bfseries\color{blue!40!green!20!violet!40!},
}
\lstdefinestyle{txt}{
    style=shared,
}
\lstdefinestyle{vimrc}{
    stringstyle=\color{gray},
    breaklines=true, 
}

\lstset{
	stringstyle=\color{gray},
	breaklines=true, 
}


\title{Competitive Programmer’s CodeBook}
\author{MIST\_EaglesExpr\\[2cm] Syed Mafijul Islam, 202214105\\
Md. Tanvin Sarkar Pallab, 202214062\\
Shihab Ahmed, 202314049
}

\setlength{\columnsep}{20pt} 
\setlength{\columnseprule}{0.1pt} 

\usepackage[hidelinks]{hyperref}
\begin{document}
\maketitle
\thispagestyle{empty}

\newpage
\tableofcontents
\newpage

\begin{multicols}{2}

\section{Useful Tips}
\begin{description}
    \item[Big Integer C++] \_\_int128\_t
    \item[C++ FastIO] \hfill \break
    ios::sync\_with\_stdio(false);
    cin.tie(nullptr);
    \item[Python FastIO] \hfill \break
    import sys; \\
    input = sys.stdin.readline
    \item[Integer - Binary Conversion in C++] \hfill
    \begin{lstlisting}
bitset<size>(val).to_string();
(int)bitset<size>(val).to_ulong();
    \end{lstlisting}
    \item[Input From File] \hfill \break
    freopen("input.txt", "r", stdin);
    \item[Python Array Input] \hfill \break
    list(map(int, input().split()))
    \item [Vim Setup] \hfill
    \lstinputlisting[style=vimrc]{.vimrc}
\end{description}

\section{Formula}
\subsection{Area Formula}
\begin{description}
    \item[Rectangle] 
        \begin{math}
            Area = length * width
        \end{math}
    \item[Square] 
        \begin{math}
            Area = Side * Side
        \end{math}
    \item[Triangle] 
        \begin{math}
            Area = \frac{1}{2} * length * width
        \end{math}
    \item[Circle] 
        \begin{math}
            Area = \pi * radius^2
        \end{math}
    \item[Parallelogram] 
        \begin{math}
            Area = base * height
        \end{math}
    \item[Pyramid Base] 
        \begin{math}
            Area = \frac{1}{2} * base * slantHeight
        \end{math}
    \item[Polygon] \hfill
        \begin{description}
            \item[a] 
            \begin{math}
                Area = \frac{1}{2}|\sum_{n=1}^{n-1}(x_iy_{i+1})|
            \end{math}

            \item[b] 
            \begin{math}
                Area = a + \frac{b}{2} - 1
            \end{math}
            (for int coordinates). Here $a=$ int points inside polygon and $b=$ int points outside polygon. 
        \end{description}
        
\end{description}
\subsection{Perimeter Formulas}
\begin{description}
    \item[Rectangle] 
        \begin{math}
            Perimeter = 2 *(length + width)
        \end{math}
    \item[Square] 
        \begin{math}
            Perimeter = 4 * side
        \end{math}
    \item[Triangle] 
        \begin{math}
            Perimeter = 4 * side
        \end{math}
    \item[Circle] 
        \begin{math}
            Perimeter = 2 * \pi * radius
        \end{math}
        
\end{description}

\subsection{Volume Formula}
\begin{description}
    \item[Cube] 
        \begin{math}
            Volume = side ^ 3
        \end{math}
    \item[Rect Prism] 
        \begin{math}
            Volume = length * width * height
        \end{math}
    \item[Cylinder] 
        \begin{math}
            Volume = \pi * radius ^2 *height
        \end{math}
    \item[Sphere] 
        \begin{math}
            Volume = \frac{4}{3}*\pi*radius^3
        \end{math}
    \item[Pyramid] 
        \begin{math}
            Volume = \frac{1}{3} *baseArea *height
        \end{math}
\end{description}

\subsection{Surface Area Formula}
\begin{description}
    \item[Cube] 
        \begin{math}
            Surface Area = 6 *side^2
        \end{math}
    \item[Rectangle Prism] 
        \begin{math}
            Surface Area = 2 *(length*width+length*height+width*height)
        \end{math}
    \item[Cylinder] 
        \begin{math}
            Surface Area = 2*\pi*radius*(radius+height)
        \end{math}
    \item[Sphere] 
        \begin{math}
            Surface Area = 4*\pi*radius^2
        \end{math}
    \item[Pyramid] 
        \begin{math}
            Surface Area = base area + \frac{1}{2}*perimeterOfBase * slantHeight
        \end{math}
\end{description}

\subsection{Triangles}
\begin{description}
    \item[Side Lengths] \(a, b, c \)
    \item[Semi Perimeter] \( p = \frac{a+b+c}{2} \)
    \item[Area] \(A = \sqrt{p(p-a)(p-b)(p-c)}\)
    \item[Circumstance] \(R =  \frac{abc}{4A} \)
    \item[In Radius] \(r = \frac{A}{p}\) 
\end{description}
\subsection{Summation Of Series}
\begin{itemize}
    \item \(c^k + c^{k+1} +...+ c^n = c^{n+1} - c^k\)
    \item \(1+2+3+...+n=\frac{n(n+1)}{2}\)
    \item \(1^2+2^2+3^2+...+n^2=\frac{n(n+1)(2n+1)}{6}\)
    \item \(1^3+2^3+3^3+...+n^3=(\frac{n(n+1)}{2})^2\)
\end{itemize}

\subsection{Miscellaneous}
\begin{itemize}
	\item \(2^{100} = 2^{50} * 2^{50}\)
	\item \( 
        \begin{bmatrix}
            F_n\\
            F_{n-1}
        \end{bmatrix} = 
        \begin{bmatrix}
            1 & 1\\
            1 & 0
        \end{bmatrix} ^ {n-1}
        \begin{bmatrix}
            F_1\\
            F_0
        \end{bmatrix}
    \)
    \item \(
        logn! = log1 +log2+...+logn
    \)
    \item \(
    	gcd(a, b) = gcd(a-b, b)
    \)
    \item Number of occurrence of a prime number $p$ in $n!$ is \(
    	\lfloor \frac{n}{p} \rfloor +	
    	\lfloor \frac{n}{p^2} \rfloor + 
    	\lfloor \frac{n}{p^3} \rfloor + ... 0
    \)
    \item Formula of Catalan number is \(
    	\begin{pmatrix}
    		2n\\
    		n
    	\end{pmatrix}
    	 -
    	\begin{pmatrix}
    		2n\\
    		n - 1
    	\end{pmatrix}
    \)
    \item Number of divisors of $p^xq^y$ where $p$ and $q$ are prime is \(
    	(x + 1) * (y + 1)
    \)
    \item Sum of divisors of $p^xq^y$ where $p$ and $q$ are prime is \(
    	(1 + p + p^2 + ... + p^x) (1 + q + q^2 + ... + q^y)
    \)
\end{itemize}

\section{Graph Theory}
All about graph.

\subsection{BFS}
\lstinputlisting[style=cpp]{code/Graph Theory/bfs.cc}

\subsection{DFS}
\lstinputlisting[style=cpp]{code/Graph Theory/dfs.cc}

\subsection{Dijkstra Algorithm}
\lstinputlisting[style=cpp]{code/Graph Theory/Dijkstra.cc}

\subsection{Bellman Ford}
\lstinputlisting[style=cpp]{code/Graph Theory/BellmanFord.cc}

\subsection{Floyed Warshall Algorithm}
\lstinputlisting[style=cpp]{code/Graph Theory/FloyedWarshall.cc}

\subsection{Kruskal Algorithm (MST \allowbreak)}
\lstinputlisting[style=cpp]{code/Graph Theory/Kruskal.cc}

\subsection{Prims Algorithm (MST \allowbreak)}
\lstinputlisting[style=cpp]{code/Graph Theory/Prims.cc}

\subsection{Strongly Connected Components}
\lstinputlisting[style=cpp]{code/Graph Theory/SCC.cc}

\subsection{LCA}
\lstinputlisting[style=cpp]{code/Graph Theory/LCA.cc}

\subsection{Max Flow}
\lstinputlisting[style=cpp]{code/Graph Theory/MaxFlow.cc}

\section{Data Structures}
Different Data Structures.

\subsection{Segment Tree}
\lstinputlisting[style=cpp]{code/Data Structures/SegmentTree.cc}

\subsection{Segment Tree Lazy}
\lstinputlisting[style=cpp]{code/Data Structures/SegmentTreeLazy.cc}

\subsection{Fenwick Tree}
\lstinputlisting[style=cpp]{code/Data Structures/FenWickTree.cc}

\subsection{Disjoint Set}
\lstinputlisting[style=cpp]{code/Data Structures/DisjointSet.cc}

\subsection{TRIE}
\lstinputlisting[style=cpp]{code/Data Structures/TRIE.cc}

\subsection{Set Balancing}
\lstinputlisting[style=cpp]{code/Data Structures/SetBalancing.cc}

\section{Algorithms}
All about algorithms.

\subsection{KMP}
\lstinputlisting[style=cpp]{code/Algorithms/KMP.cc}

\subsection{Monotonic Stack (Immediate \allowbreak Small)}
\lstinputlisting[style=cpp]{code/Algorithms/MonotonicStack_ImmediateSmall.cc}

\subsection{Kadane’s Algorithm}
\lstinputlisting[style=cpp]{code/Algorithms/Kadane.cc}

\subsection{2D Prefix Sum}
\lstinputlisting[style=cpp]{code/Algorithms/2DPrefixSum.cc}

\section{Number Theory}
All about math.

\subsection{nCr}
\lstinputlisting[style=cpp]{code/Number Theory And Maths/nCr.cc}

\subsection{Power}
\lstinputlisting[style=cpp]{code/Number Theory And Maths/Power.cc}

\subsection{Miller Rabin}
\lstinputlisting[style=cpp]{code/Number Theory And Maths/MillerRabin.cc}

\subsection{Sieve}
\lstinputlisting[style=cpp]{code/Number Theory And Maths/Sieve.cc}

\subsection{Bitset Sieve}
\lstinputlisting[style=cpp]{code/Number Theory And Maths/BitsetSieve.cc}

\subsection{Divisors}
\lstinputlisting[style=cpp]{code/Number Theory And Maths/Divisor.cc}

\subsection{Euler's Totient Phi Function}
\lstinputlisting[style=cpp]{code/Number Theory And Maths/Phi.cc}

\subsection{Log a base b}
\lstinputlisting[style=cpp]{code/Number Theory And Maths/logab.cc}

\subsection{Count 1's from 0 to n}
\lstinputlisting[style=cpp]{code/Number Theory And Maths/cnt1sIn0ToN.cc}

\subsection{Primes Upto 1e9}
\lstinputlisting[style=cpp]{code/Number Theory And Maths/SieveUpto10e9.cc}


\section{Dynamic Programming}
\subsection{LCS (Longest \allowbreak Common Subsequence)}
\lstinputlisting[style=cpp]{code/Dynamic Programming/LCS.cc}

\subsection{LIS (Longest \allowbreak Increasing Subsequence)}
\lstinputlisting[style=cpp]{code/Dynamic Programming/LIS.cc}

\subsection{SOS (Sum \allowbreak Of Subsets)}
\lstinputlisting[style=cpp]{code/Dynamic Programming/SOS.cc}

\end{multicols}
\end{document}