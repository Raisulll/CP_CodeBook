\documentclass[10pt, a4paper,twocolumn]{article}
\usepackage[a4paper, left=1in, right=1in, top=1in, bottom=1in]{geometry}
\usepackage{graphicx}
\usepackage{mdframed}
% \usepackage{multicol}
\usepackage{listings}
\usepackage{xcolor} 
\usepackage{amsmath}
\usepackage{fancyhdr}      
\usepackage[hidelinks]{hyperref}     
\pagestyle{fancy}   

\fancyhf{}

\fancyhead[L]{Military Institute of Science and Technology MIST\_CodeCrafters} 
\fancyhead[R]{\thepage} 

\lstdefinestyle{shared}{
    belowcaptionskip=1\baselineskip,
    breaklines=true, 
    showstringspaces=false,
    basicstyle=\ttfamily\small,
    tabsize=2,
}
\lstdefinestyle{cpp}{
    style=shared,
    language=C++,
    commentstyle=\itshape\color{green!50!black},
    keywordstyle=\bfseries\color{blue},
    stringstyle=\color{orange},
    numberstyle=\tiny\color{gray},
}
\lstdefinestyle{java}{
    style=shared,
    language=Java,
    keywordstyle=\bfseries\color{red!22!green!28!blue!50!},
    commentstyle=\itshape\color{black!60!},
    identifierstyle=\bfseries\color{red!19!green!42!blue!39!},
    stringstyle=\bfseries\color{blue!40!green!20!violet!40!},
}
\lstdefinestyle{py}{
    style=shared,
    language=Python,
    keywordstyle=\bfseries\color{red!22!green!28!blue!50!},
    commentstyle=\itshape\color{black!60!},
    identifierstyle=\bfseries\color{red!19!green!42!blue!39!},
    stringstyle=\bfseries\color{blue!40!green!20!violet!40!},
}
\lstdefinestyle{txt}{
    style=shared,
}

\lstset{
	stringstyle=\color{gray},
	breaklines=true, 
}



\setlength{\columnsep}{20pt} 
\setlength{\columnseprule}{0.1pt} 
\begin{document}
\newpage
\tableofcontents
\newpage

%\begin{multicols}{2}

\section{Useful Tips}
\begin{description}
    \subsection{Run with key in VS Code}
    Add the following code in \texttt{keybindings.json} file in VS Code, to run onpen git bash terminal, in.txt and out.txt file and press \texttt{f5} to run the code.
    \begin{lstlisting}[style=txt]
    {
      "key": "f5",
      "command": "workbench.action.terminal.sendSequence",
      "args": {
        "text": "g++ ${fileBasenameNoExtension}.cpp -o ${fileBasenameNoExtension} && ./${fileBasenameNoExtension} < in.txt > out.txt\n"
      }
    }
    % now i want to add comment to 
    \end{lstlisting}
    \item[Run CPP File] \hfill \break
    \texttt{g++ .\textbackslash test.cpp -o test \&\& .\textbackslash test}
    \item[Run CPP File with Input File] \hfill \break
    \texttt{g++ .\textbackslash test.cpp -o test \&\& .\textbackslash test < in.txt > out.txt}
    % \texttt{g++ .\test.cpp -o test && .\test < in.txt > out.txt}
    \item[Big Integer C++] \_\_int128\_t
    \item[C++ FastIO] \hfill \break
    ios::sync\_with\_stdio(false);
    cin.tie(nullptr);
    \item[Python FastIO] \hfill \break
    import sys; \\
    input = sys.stdin.readline
    \item[Integer - Binary Conversion in C++] \hfill
    \begin{lstlisting}
bitset<size>(val).to_string();
(int)bitset<size>(val).to_ulong();
    \end{lstlisting}
    \item[Input From File] \hfill \break
    freopen("input.txt", "r", stdin);
    \item[Python Array Input] \hfill \break
    list(map(int, input().split()))
\end{description}

\section{Formula}
\subsection{Area Formula}
\begin{description}
    \item[Rectangle] 
        \begin{math}
            Area = length * width
        \end{math}
    \item[Square] 
        \begin{math}
            Area = Side * Side
        \end{math}
    \item[Triangle] 
        \begin{math}
            Area = \frac{1}{2} * length * width
        \end{math}
    \item[Circle] 
        \begin{math}
            Area = \pi * radius^2
        \end{math}
    \item[Parallelogram] 
        \begin{math}
            Area = base * height
        \end{math}
    \item[Pyramid Base] 
        \begin{math}
            Area = \frac{1}{2} * base * slantHeight
        \end{math}
    \item[Polygon] \hfill
        \begin{description}
            \item[a] 
            \begin{math}
                Area = \frac{1}{2}|\sum_{n=1}^{n-1}(x_iy_{i+1})|
            \end{math}

            \item[b (Pick's formula)] 
            \begin{math}
                Area = a + \frac{b}{2} - 1
            \end{math}
            (for int coordinates). Here $a=$ int points inside polygon and $b=$ int points outside polygon. 
        \end{description}
        
\end{description}
\subsection{Perimeter Formulas}
\begin{description}
    \item[Rectangle] 
        \begin{math}
            Perimeter = 2 *(length + width)
        \end{math}
    \item[Square] 
        \begin{math}
            Perimeter = 4 * side
        \end{math}
    \item[Triangle] 
        \begin{math}
            Perimeter = 4 * side
        \end{math}
    \item[Circle] 
        \begin{math}
            Perimeter = 2 * \pi * radius
        \end{math}
        
\end{description}

\subsection{Volume Formula}
\begin{description}
    \item[Cube] 
        \begin{math}
            Volume = side ^ 3
        \end{math}
    \item[Rect Prism] 
        \begin{math}
            Volume = length * width * height
        \end{math}
    \item[Cylinder] 
        \begin{math}
            Volume = \pi * radius ^2 *height
        \end{math}
    \item[Sphere] 
        \begin{math}
            Volume = \frac{4}{3}*\pi*radius^3
        \end{math}
    \item[Pyramid] 
        \begin{math}
            Volume = \frac{1}{3} *baseArea *height
        \end{math}
\end{description}

\subsection{Surface Area Formula}
\begin{description}
    \item[Cube] 
        \begin{math}
            Surface Area = 6 *side^2
        \end{math}
    \item[Rectangle Prism] 
        \begin{math}
            Surface Area = 2 *(length*width+length*height+width*height)
        \end{math}
    \item[Cylinder] 
        \begin{math}
            Surface Area = 2*\pi*radius*(radius+height)
        \end{math}
    \item[Sphere] 
        \begin{math}
            Surface Area = 4*\pi*radius^2
        \end{math}
    \item[Pyramid] 
        \begin{math}
            Surface Area = base area + \frac{1}{2}*perimeterOfBase * slantHeight
        \end{math}
\end{description}

\subsection{Triangles}
\begin{description}
    \item[Side Lengths] \(a, b, c \)
    \item[Semi Perimeter] \( p = \frac{a+b+c}{2} \)
    \item[Area] \(A = \sqrt{p(p-a)(p-b)(p-c)}\)
    \item[Circumstance] \(R =  \frac{abc}{4A} \)
    \item[In Radius] \(r = \frac{A}{p}\) 
\end{description}
\subsection{Summation Of Series}
\begin{itemize}
    \item \(c^k + c^{k+1} +...+ c^n = c^{n+1} - c^k\)
    \item \(1+2+3+...+n=\frac{n(n+1)}{2}\)
    \item \(1^2+2^2+3^2+...+n^2=\frac{n(n+1)(2n+1)}{6}\)
    \item \(1^3+2^3+3^3+...+n^3=(\frac{n(n+1)}{2})^2\)
    \item \(1^4+2^4+3^4+...+n^4=\frac{n(n+1)(2n+1)(3n^2+3n-1)}{30}\)

    % \item [Sum of first n odd numbers] \(n^2\)
\end{itemize}
    {    Sum of first n odd numbers} \(n^2\)
\subsection{Logarithmic Basics}
\begin{itemize}
    \item \(\log_b 1 = 0\)
    \item \(\log_b b = 1\)
    \item \(\log_b (AB) = \log_b A + \log_b B\)
    \item \(\log_b \left(\frac{A}{B}\right) = \log_b A - \log_b B\)
    \item \(\log_b A^x = x \log_b A\)
    \item \(\log_a c = \log_a b \cdot \log_b c\)
    \item \(b^{\log_b a} = a\)
    \item \(x \log_b y = y \log_b x\)
    \item \(\log_a b = \frac{1}{\log_b a}\)
    \item \(\log_a x = \frac{\log_b x}{\log_b a}\)
\end{itemize}

\subsection{Series}

\subsubsection{Catalan Series} 
Series: \(1, 1, 2, 5, 14, 42, 132, 429, \dots\) \\
Equation: \(C_n = \sum_{k=0}^{n-1} C_k \cdot C_{n-1-k}\)

\subsubsection{Arithmetic Series} 
\(a_n = a + (n - 1) \cdot d\) \\
\(S_n = \frac{n}{2} \cdot (2a + (n - 1) \cdot d)\)

\subsubsection{Geometric Series} 
\(a_n = a \cdot r^{n-1}\) \\
\(S_n = \frac{a(1 - r^n)}{1 - r} \quad \text{for} \quad r \neq 1\)

\subsubsection{Derangement Series} 
% Series: \(0, 1, 2, 9, 44, 265, 1854, 14833, 133496, 1334961, 14684570, \dots\) \\ make this line in two line
Series: \(0, 1, 2, 9, 44, 265, 1854, 14833, 133496,\) \\
\(1334961, 14684570, \dots\) \\
\(D_n = (n - 1)D_{n - 2} + (n - 1)D_{n - 1}\)

\subsubsection{nth Fibonacci Golden Ratio} 
\[
f_n = \frac{\left( \left( \frac{1 + \sqrt{5}}{2} \right)^n - \left( \frac{1 - \sqrt{5}}{2} \right)^n \right)}{\sqrt{5}}
\]




\subsection{Miscellaneous}
\begin{itemize}
	\item \(2^{100} = 2^{50} * 2^{50}\)
	\item \( 
        \begin{bmatrix}
            F_n\\
            F_{n-1}
        \end{bmatrix} = 
        \begin{bmatrix}
            1 & 1\\
            1 & 0
        \end{bmatrix} ^ {n-1}
        \begin{bmatrix}
            F_1\\
            F_0
        \end{bmatrix}
    \)
    \item \(
        logn! = log1 +log2+...+logn
    \)
    \item Number of occurrence of a prime number $p$ in $n!$ is \(
    	\lfloor \frac{n}{p} \rfloor +	
    	\lfloor \frac{n}{p^2} \rfloor + 
    	\lfloor \frac{n}{p^3} \rfloor + ... + 0
    \)
    \item Number of divisors of $p^xq^y$ where $p$ and $q$ are prime is \(
    	(x + 1) * (y + 1)
    \)
    \item Sum of divisors of $p^xq^y$ where $p$ and $q$ are prime is \(
    	(1 + p + p^2 + ... + p^x) (1 + q + q^2 + ... + q^y)
    \)
   \item Golden Ratio \(
   	\Phi \approx 1.618034
   \)
   \item $n$th Fibonacci number \(
   	F_n = \frac{\Phi n - ( 1 - \Phi)n} {\sqrt{5}}
   \)
   \item \(n(A \cup B) = n(A) + n(B) - n(A \cap B)\)
   \item If \(A \cap B = \emptyset\), then \(n(A \cup B) = n(A) + n(B)\)
   \item \(n(A - B) + n(A \cap B) = n(A)\)
   \item \(n(B - A) + n(A \cap B) = n(B)\)
   \item \(n(A \cup B) = n(A - B) + n(A \cap B) + n(B - A)\)
   \item \(n(A \cup B \cup C) = n(A) + n(B) + n(C) - n(A \cap B) - n(B \cap C) - n(C \cap A) + n(A \cap B \cap C)\)
   \item Area of a regular polygon: \(\frac{n a^2 \cot\left(\frac{180}{n}\right)}{4}\)
   \item Apex point angle of a regular polygon: \(\left( \frac{2n - 4}{n} \right) \times 90^\circ\)
\end{itemize}

\section{Graph Theory}
All about graph.

\subsection{BFS}
\lstinputlisting[style=cpp]{code/Graph Theory/bfs.cc}

\subsection{DFS}
\lstinputlisting[style=cpp]{code/Graph Theory/dfs.cc}

\subsection{Dijkstra Algorithm}
\lstinputlisting[style=cpp]{code/Graph Theory/Dijkstra.cc}

\subsection{Bellman Ford}
\lstinputlisting[style=cpp]{code/Graph Theory/BellmanFord.cc}

\subsection{Floyed Warshall Algorithm}
\lstinputlisting[style=cpp]{code/Graph Theory/FloyedWarshall.cc}

\subsection{Kruskal Algorithm (MST \allowbreak)}
\lstinputlisting[style=cpp]{code/Graph Theory/Kruskal.cc}

\subsection{Prims Algorithm (MST \allowbreak)}
\lstinputlisting[style=cpp]{code/Graph Theory/Prims.cc}

\subsection{Strongly Connected Components}
\lstinputlisting[style=cpp]{code/Graph Theory/SCC.cc}

\subsection{LCA}
\lstinputlisting[style=cpp]{code/Graph Theory/LCA.cc}

\subsection{Max Flow}
\lstinputlisting[style=cpp]{code/Graph Theory/MaxFlow.cc}

\subsection{Cycle Detection in Directed Graph}
\lstinputlisting[style=cpp]{code/Graph Theory/CycleDetection1.cc}

\subsection{Cycle Detection in Undirected Graph}
\lstinputlisting[style=cpp]{code/Graph Theory/CycleDetection2.cc}

\subsection{Bipartite Graph Check}
\lstinputlisting[style=cpp]{code/Graph Theory/BipartiteGraphTest.cc}

\section{Data Structures}
Different Data Structures.

\subsection{Segment Tree}
\lstinputlisting[style=cpp]{code/Data Structures/SegmentTree.cc}

\subsection{Segment Tree Lazy}
\lstinputlisting[style=cpp]{code/Data Structures/SegmentTreeLazy.cc}

\subsection{Fenwick Tree}
\lstinputlisting[style=cpp]{code/Data Structures/FenWickTree.cc}

\subsection{Disjoint Set}
\lstinputlisting[style=cpp]{code/Data Structures/DisjointSet.cc}

\subsection{TRIE}
\lstinputlisting[style=cpp]{code/Data Structures/TRIE.cc}

\subsection{Set Balancing}
\lstinputlisting[style=cpp]{code/Data Structures/SetBalancing.cc}

\subsection{Ordered Set}
\lstinputlisting[style=cpp]{code/Data Structures/OrderedSet.cc}

\section{Algorithms}

\subsection{KMP}
\lstinputlisting[style=cpp]{code/Algorithms/KMP.cc}

\subsection{Monotonic Stack (Immediate \allowbreak Small)}
\lstinputlisting[style=cpp]{code/Algorithms/MonotonicStack_ImmediateSmall.cc}

\subsection{Kadane’s Algorithm}
\lstinputlisting[style=cpp]{code/Algorithms/Kadane.cc}

\subsection{2D Prefix Sum}
\lstinputlisting[style=cpp]{code/Algorithms/2DPrefixSum.cc}

\subsection{Next Greater Element}
\lstinputlisting[style=cpp]{code/Algorithms/NextGreaterElement.cc}

\section{Number Theory}
\subsection{Prime Numbers Under 1000}

The prime numbers under 1000 are:

\begin{itemize}
    \item 2, 3, 5, 7, 11, 13, 17, 19, 23, 29, 31, 37, 41, 43, 47, 53, 59, 61, 67, 71, 73, 79, 83, 89, 97, 
    101, 103, 107, 109, 113, 127, 131, 137, 139, 149, 151, 157, 163, 167, 173, 179, 181, 191, 193, 197, 199, 
    211, 223, 227, 229, 233, 239, 241, 251, 257, 263, 269, 271, 277, 281, 283, 293, 307, 311, 313, 317, 331, 
    337, 347, 349, 353, 359, 367, 373, 379, 383, 389, 397, 401, 409, 419, 421, 431, 433, 439, 443, 449, 457, 
    461, 463, 467, 479, 487, 491, 499, 503, 509, 521, 523, 541, 547, 557, 563, 569, 571, 577, 587, 593, 599, 601, 607, 613, 617, 619, 631, 641, 643, 647, 653, 659, 
    661, 673, 677, 683, 691, 701, 709, 719, 727, 733, 739, 743, 751, 757, 761, 769, 773, 787, 797, 809, 811, 821,
    823, 827, 829, 839, 853, 857, 859, 863, 877, 881, 883, 887, 907, 911, 919, 929, 937, 941, 947, 953, 967, 
    971, 977, 983, 991, 997
\end{itemize}

\subsection{Divisibility Rules for Numbers}
\begin{itemize}
    \item[3] Sum of digits divisible by $3$.
    \item[4] Last two digits divisible by $4$.
    \item[5] Ends in $0$ or $5$.
    \item[6] Divisible by both $2$ and $3$.
    \item[7] Double the last digit, subtract it from the rest; check if the result is divisible by $7$. Repeat if necessary.
    \item[8] Last three digits divisible by $8$.
    \item[9] Sum of digits divisible by $9$.
    \item[11] Difference between sums of digits in odd and even places divisible by $11$.
\end{itemize}


\subsection{Divisor Count}
\lstinputlisting[style=cpp]{code/Number Theory And Maths/DivisorCount.cc}

\subsection{Leap Year}
\lstinputlisting[style=cpp]{code/Number Theory And Maths/LeapYear.cc}

\subsection{Number of Leap Year in Between}
\lstinputlisting[style=cpp]{code/Number Theory And Maths/NumberOfLeapYearInBetween.cc}
\subsection{Binary Exponentiation: (a\^ b)}
\lstinputlisting[style=cpp]{code/Number Theory And Maths/BinaryExponentiation1.cc}
\subsection{Binary Exponentiation: (a\^ b\^ c)}
\lstinputlisting[style=cpp]{code/Number Theory And Maths/BinaryExponentiation2.cc}
\subsection{Power}
\lstinputlisting[style=cpp]{code/Number Theory And Maths/power.cc}
\subsection{nCr}
\lstinputlisting[style=cpp]{code/Number Theory And Maths/nCr.cc}

\subsection{Check Prime -O(sqrt(n))}
\lstinputlisting[style=cpp]{code/Number Theory And Maths/PrimeCheck.cc}
\subsection{Prime Factorization}
\lstinputlisting[style=cpp]{code/Number Theory And Maths/PrimeFactorization.cc}
\subsection{Miller Rabin}
\lstinputlisting[style=cpp]{code/Number Theory And Maths/MillerRabin.cc}

\subsection{Sieve}
\lstinputlisting[style=cpp]{code/Number Theory And Maths/Sieve.cc}
\subsection{BitSet Sieve}
\lstinputlisting[style=cpp]{code/Number Theory And Maths/BitSetSieve.cc}

\subsection{Segment Sieve}
\lstinputlisting[style=cpp]{code/Number Theory And Maths/SegmentSieve.cc}

\subsection{Nth Prime}
\lstinputlisting[style=cpp]{code/Number Theory And Maths/NthPrime.cc}

\subsection{Divisors}
\lstinputlisting[style=cpp]{code/Number Theory And Maths/Divisor.cc}

% \subsection{Euler's Totient Phi Function}
% \lstinputlisting[style=cpp]{code/Number Theory And Maths/Phi.cc}

\subsection{Log a base b}
\lstinputlisting[style=cpp]{code/Number Theory And Maths/logab.cc}

\subsection{Count 1's from 0 to n}
\lstinputlisting[style=cpp]{code/Number Theory And Maths/cnt1sIn0ToN.cc}

\subsection{Primes Upto 1e9}
\lstinputlisting[style=cpp]{code/Number Theory And Maths/SieveUpto10e9.cc}
\subsection{Extd GCD}
\lstinputlisting[style=cpp]{code/Number Theory And Maths/extd_gcd.cc}

\subsection{Factorial Mod}
\lstinputlisting[style=cpp]{code/Number Theory And Maths/Factorial_Mod.cc}
\subsection{Modular Operations}
\lstinputlisting[style=cpp]{code/Number Theory And Maths/Modular_operation.cc}

\subsection{10 base to M base conversion}
\lstinputlisting[style=cpp]{code/Number Theory And Maths/10BaseToMBase.cc}

\subsection{M base to 10 base conversion}
\lstinputlisting[style=cpp]{code/Number Theory And Maths/Mto10Base.cc}

\subsection{Matrix Exponentiation}
\lstinputlisting[style=cpp]{code/Number Theory And Maths/MatrixExponential.cc}



\section{Dynamic Programming}
\subsection{LCS (Longest \allowbreak Common Subsequence)}
\lstinputlisting[style=cpp]{code/Dynamic Programming/LCS.cc}

\subsection{LIS (Longest \allowbreak Increasing Subsequence)}
\lstinputlisting[style=cpp]{code/Dynamic Programming/LIS.cc}

\subsection{SOS (Sum \allowbreak Of Subsets)}
\lstinputlisting[style=cpp]{code/Dynamic Programming/SOS.cc}


\section{Strings}
\subsection{Double Hashing}
Need power() from \ref{power}
\lstinputlisting[style=cpp]{code/Strings/hashing.cc}

\subsection{Large Number Multiplication}
\lstinputlisting[style=cpp]{code/Strings/MultiplyLargeNumber.cc}

\subsection{Large Number Summation}
\lstinputlisting[style=cpp]{code/Strings/SumofLargeNumber.cc}

\subsection{Large Number Division}
\lstinputlisting[style=cpp]{code/Strings/DivisionLargeNumber.cc}

\subsection{Large Number Subtraction}
\lstinputlisting[style=cpp]{code/Strings/SubtractionLargeNumber.cc}

\section{Stress Testing}
\subsection{Bash Stress File}
\lstinputlisting[style=cpp]{code/Stress Test/stress.sh}

\subsection{C++ Generator File}
\lstinputlisting[style=cpp]{code/Stress Test/gen.cc}

\section{Game Theory}
\subsection{Nim Game}
The current player has a winning strategy if and only if the xor-sum of the pile sizes is non-zero.
\subsection{Miser Nim}
-Last player to remove stones loses.
-Winning state if xor-sum of pile sizes is non-zero.
-Exception: Each pile has one stone only.
-Winning strategy: If there is only one pile of size greater than one,take all or all but one from that pile leaving an odd number one-size piles. Otherwise, same as normal nim.
\subsection{Grundy Game}
\lstinputlisting[style=cpp]{code/Game Theory/GrundysGame.cc}

%\end{multicols}
\end{document}